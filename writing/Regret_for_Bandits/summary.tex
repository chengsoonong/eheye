\documentclass{article}
\usepackage[utf8]{inputenc}
\usepackage{biblatex}
\usepackage{amsmath}
\usepackage{amssymb}
\usepackage[mathscr]{eucal}
\usepackage[dvipsnames]{xcolor}
\DeclareMathOperator*{\argmax}{argmax}
\usepackage{amsthm}
\usepackage{tcolorbox}
\usepackage{graphicx}
\usepackage{tikz}
\usepackage{textcomp}

%----------------------------------------------------------------------
\topmargin -.5in
\textheight 9in
\oddsidemargin -.25in
\evensidemargin -.25in
\textwidth 7in

\newtheorem{theo}{Theorem}

\title{Regret for Bandits}
\author{u6015325 }
\date{August 2019}
\bibliography{ref.bib}

\begin{document}

\maketitle

There are several different definitions about the evaluation metrics for multi-armed bandits problems. In this article, we will go through the types of bandits problems and the different goal settings of literature. There are two families of bandit algorithms:

\begin{enumerate}
    \item Exploration-exploitation balance. The usual goal is to maximize the cumulative rewards.
    \item Explore then commit i.e. Best arm identification problem or pure exploration problem \cite{audibert2010best}. Assume that after a given number of pulls, the forecaster is asked to output a recommended arm. He is then only evaluated by the average payoff of his recommended arm. The goal is to minimize the probability that the recommendation is not the optimal one. 
\end{enumerate}

In the rest of the article, we will focus on the UCB-type algorithms for the first group. 

\section{Regret based on expected value}

There common choices of the metrics are \footnote{In this article, we use the same notation as https://github.com/Mengyanz/paper/tree/master/writing/HazardUCB}:

\begin{enumerate}
    \item cumulative regret, which is defined as s the loss in the cumulative rewards resulting from choosing sub-optimal leaves instead of an optimal one:
    \begin{align}
        \label{equ: regret}
        R_{N} \stackrel{\operatorname{def}}{=} \sum_{t=1}^{N} X_{\ast, t}-\sum_{t=1}^{N} \sum_{i = 1}^{K} X_{i, T_i(t)} \mathbb{I}\{A_t = i\},
    \end{align}
    where $X_{i,T_i(t)}$ as the reward i.i.d. sampled from $P_i$ in round $t$ (when arm $i$ has been played $T_i(t)$ times),  $\mathbb{I}\{A_t = i\}$ is the indicator function which returns 1 is $A_t = i$ is true, otherwise returns 0, i.e.  $\sum_{i = 1}^{K} X_{i, T_i(t)} \mathbb{I}\{A_t = i\}$ is the reward sampled at the $t$ round.  
    \item cumulative pseudo-regret, which is defined on the true mean of arm distributions, 
    \begin{align}
        \label{equ: pseudo regret}
        \overline{R}_{N} & \stackrel{\text { def }}{=}
        N \mu^{*}-\sum_{t=1}^{N} \sum_{i = 1}^{K} \mu_i \mathbb{I}\{A_t = i\}\\
        &= N \mu^{*}-\sum_{i = 1}^{K} \mu_i \sum_{t=1}^{N}   \mathbb{I}\{A_t = i\}\\
        \label{equ: pseudo regret simplified}
        &= N \mu^{*}-\sum_{i = 1}^{K} \mu_i T_i(N)
    \end{align}
    The difference between the regret and the pseudo-regret comes from the randomness of the rewards.
\end{enumerate}

\subsection{Prove $\mathbb{E}[R_N] = \mathbb{E}[\overline{R}_{N}]$}

\begin{theo} (Wald’s Identity). If $Y_1, Y_2, Y_3, ...$ be i.i.d. with finite mean, and $N$ is a stopping time with $\mathbb{E}[N] < \infty$, then $\mathbb{E}\left[Y_{1}+\cdots+Y_{N}\right]=\mathbb{E}\left[Y_{1}\right] \mathbb{E}[N]$.
\end{theo}

Thanks to Wald's theorem, the regret and pseudo-regret have the same expectation, 

\begin{align}
    \mathbb{E}[R_N] &= \mathbb{E}\left[\sum_{t=1}^{N} X_{\ast, t}-\sum_{t=1}^{N} \sum_{i = 1}^{K} X_{i, T_i(t)} \mathbb{I}\{A_t = i\}\right]\\
    &= \mathbb{E}\left[\sum_{t=1}^{N} X_{\ast, t}\right]- \mathbb{E}\left[\sum_{t=1}^{N} \sum_{i = 1}^{K} X_{i, T_i(t)} \mathbb{I}\{A_t = i\}\right] & \text{By expectation linearity}\\
    &= \mu_\ast N - \sum_{i = 1}^{K} \mu_i \mathbb{E}[T_i(N)] 
    & \text{By Wald's Theorem}\\
    \mathbb{E}[\overline{R}_N] &= \mathbb{E}\left[N \mu^{*}-\sum_{i = 1}^{K} \mu_i T_i(N)\right]\\
    &= \mu_\ast N - \mathbb{E}\left[\sum_{i = 1}^{K} \mu_i T_i(N)\right] & \text{By expectation linearity}\\
    &= \mu_\ast N - \sum_{i = 1}^{K} \mu_i \mathbb{E}\left[T_i(N)\right] & \text{By expectation linearity}
\end{align}

\subsection{Decomposition of pseudo-regret}

The pseudo-regret can be decomposed into the weighted sum of sub-optimal draws, where the weighted is the gap $\Delta_i = \mu_\ast - \mu_i$, 

\begin{align}
    \overline{R}_N &= \sum_{i=1}^K \mu_\ast T_i(N) - \sum_{i = 1}^{K} \mu_i T_i(N) & \text{By } (\ref{equ: pseudo regret simplified}) \text{ and } \sum_{i = 1}^K T_i(N) = N\\
    &= \sum_{i=1}^K (\mu_\ast - \mu_i) T_i(N)\\
    &= \sum_{i=1}^K \Delta_i T_i(N)
\end{align}

Then from last section, we can also get the decomposed version of expected regret and pseudo-regret, 

\begin{align}
\mathbb{E}[R_N] = \mathbb{E}[\overline{R}_N] = \mathbb{E}\left[\sum_{i=1}^K \Delta_i T_i(N)\right] = \sum_{i=1}^K \Delta_i \mathbb{E}[T_i(N)]
\end{align}


\subsection{Gap between regret and pseudo-regret}

\textcite{coquelin2007bandit} stated the bound of gap between regret and pseudo-regret $|R_N - \overline{R}_N|$, using Azuma’s inequality for martingale difference sequences, with probability
at least $1 - \beta$, we have at time $N$,
\begin{align}
    \left|R_{n}-\overline{R}_{n}\right| \leq \sqrt{|\operatorname{Sub}(n)| \log (2 / \beta) / 2}
\end{align}
where $|\operatorname{Sub}(n)|$ is the cardinal of $\operatorname{Sub}(n) \stackrel{\text { def }}{=}\{t \in \left.\{1, \ldots, n\}, I_{t} \neq i^{*}\right\}$.

\section{Risk-averse Regret}

It is not obvious why the expected value is a good summary of the reward distribution. Decision makers who base their decisions on expected values are risk-neutral. A decision maker may would like to avoid risky decisions with unexpected low or negative reward. Those decision makers are called risk-averse. There is no universal definition for risk-averse regret, we show two general form to illustrate the idea of design. 
    \begin{enumerate}
         \item \textcite{cassel2018general} showed a general approach for bandits under risk criteria, under the class of \textit{Empirical Distribution Performance Measures(EDPM)}. An EDPM evaluates performance by means of a function $U$ , which maps $\hat{F} \text { to } \mathbb{R}, \text { i.e., } U(\hat{F})=\tilde{U}\left(X_{1}, \ldots, X_{t}\right)$ The expected regret and expected pseudo-regret is 
        \begin{align}
            & R_{\pi}(N) :=\mathbb{E}\left[U\left(\hat{F}_{N}^{\pi^{*}(N)}\right)-U\left(\hat{F}_{N}^{\pi}\right)\right]\\
            & \overline{R}_{\pi}(N) :=\mathbb{E}\left[U\left(F_{p^{*}}\right)-U\left(F_{N}^{\pi}\right)\right]
        \end{align}
        where $F_{N}^{\pi}=\frac{1}{N} \sum_{t=1}^{N} F^{\left(\pi_{t}\right)}=\frac{1}{N} \sum_{i=1}^{K} T_{i}(N) F^{(i)},\hat{F}_{t}\left(x_{1}, \ldots, x_{t} ; \cdot\right)=\frac{1}{t} \sum_{s=1}^{t} \mathbb{I}_{\left[x_{s}, \infty\right]}(\cdot) , \hat{F}_{t}^{\pi}(\cdot) :=\hat{F}_{t}\left(X_{1}^{\pi}, \ldots, X_{t}^{\pi} ;\right)$
        \begin{itemize}
            \item When $U$ is linear
            \begin{itemize}
                \item the expected regret and expected pseudo-regret are same. 
                \item proxy regret can be decomposed into
                \begin{align}
                    R_{\pi}(T)=\frac{1}{T} \sum_{i \neq i^{*}} \mathbb{E}\left[T_{i}(T)\right] \Delta_{i},
                \end{align}
                where $\Delta_{i}=U\left(F_{p^{*}}\right)-U\left(F^{(i)}\right)$.\\
                Note: when U is linear $U(F_T^\pi) = U(\frac{1}{T} \sum_{i=1}^K F_{(i)} T_i(T)) = \frac{1}{T} \sum_{i=1}^K U(F_{(i)}) T_i(T)$.
                \end{itemize}
            \item When $U$ is quasiconvex and is strongly stable EDPM (Definition 3 in \cite{cassel2018general}), proxy regret can be bounded by,
                \begin{align}
                    \overline{R}_{\pi}(T) \leq \frac{L}{T} \sum_{i \neq i^{*}} \mathbb{E}\left[T_{i}(T)\right]\left\|F^{\left(i^{*}\right)}-F^{(i)}\right\|
                \end{align}
        \end{itemize}
        \item \textcite{maillard2013robust} introduce the coherent risk measure for robust risk-averse stochastic bandit problems, where the empirical regret and risk-averse regret are
        \begin{align}
            & \mathfrak{R}_{T}(\lambda) \stackrel{\text { def }}{=} \sum_{i=1}^{T} X_{i, a^{\star}}-\sum_{a=1}^{A} \sum_{i=1}^{N_{T, a}^{\mathbb{Z}}} X_{i, a}=\sum_{i=N_{T, a}^{\infty}, a^{\star}+1}^{T} X_{i, a^{\star}}-\sum_{a \neq a^{\star}} \sum_{i=1}^{N_{T, a}^{\mathbb{2}}} X_{i, a}.\\
            & \overline{\mathfrak{R}}_{T}(\lambda)=\sum_{a \in \mathcal{A}}\left(\kappa_{-\lambda, \nu_{a^{*}}}-\kappa_{-\lambda, \nu_{a}}\right) \mathbb{E}\left[N_{T, a}\right].
        \end{align}
        where $\kappa_{\lambda, \nu}=\frac{1}{\lambda} \log \mathbb{E}_{\nu} \exp (\lambda X)$.
    \end{enumerate}
       

\printbibliography
\end{document}
