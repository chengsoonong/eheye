\documentclass[11pt]{article}

\usepackage[margin=1in]{geometry}
\usepackage{amssymb}
\usepackage{amsmath}
\usepackage{bm}
\usepackage{xcolor}

\def\R{\rm I\!R}
\def\x{\bm{x}}

\title{Proof of Algorithm Equality}

\begin{document}
\maketitle
% --------------------------------------------------------------------------------
%                                Quantile Estimation
% --------------------------------------------------------------------------------
            
\section{Quantile Estimation}

\subsection{Quantile}

In statistics, quantiles are the points that divide a probability distribution into even intervals.
The $q$-quantiles devide the distribution into $q$ intervals each with the same amount of data points.
And there are $q$ quantile points of the $q-$quantiles.
For example, the $2$-quantile has only one quantile point, which is the middle point of the distribution
and it divides the distribution into two even parts. This $2$-quantile point is called the median.


\subsubsection{Definition}
Generally, the $q$-quantiles have $q-1$ quantile points, and the $k$th $q$-quantile for a 
distribution $X$ is the data value such that
$$
Pr(X \leq x) \geq \frac{k}{q}
$$
and
$$
Pr(X \geq x) \geq 1 - \frac{k}{q}
$$
where $x \in X$

\subsection{Quantile Estimation and Pinball Loss}
In this paper, the estimation for $\tau$-quantile 
($\tau =  \frac{1}{q}, \frac{2}{q}, \cdots, \frac{q-1}{q}$)
is applied.
Pinball loss function is one of the approaches for the estimation for a statistical population.
\\\\
For a one-dimentional data set $X = \{x_1, x_2, \cdots, x_N\}$, 
now consider the loss function for a single data point $x$ $(i \in {1, \cdots, N})$.
Let $t := x - q$ be the difference between the real value $x$ and the estimate of quantile $q$.
$l_{\tau}(\cdot): \R \to \R_{\geq 0}$ is the loss function on $t$ such that
$$
l_\tau(t)= 
    \begin{cases}
        \tau t & t > 0\\
        -(1-\tau) t & otherwise
    \end{cases}
$$
And the $\tau$-quantile loss has the {\color{red} subgradient}:
$$
\frac {\partial l_\tau(t)}{\partial t}= 
    \begin{cases}
        \tau                & t > 0\\
        -(1-\tau)           & t < 0\\
        [\tau, -(1 - \tau)] & t = 0
    \end{cases}
$$

The overall loss for distribution $X$ with quantile estimation $q$ is
$$
L_{\tau}(q) = \sum_{x \in X} l_{\tau}(x - q)
$$
The best estimate of the $\tau$-quantile $q$ is the $q$ with minimal overall loss. 
Let $q^\ast$ be the best estimate, then we have
$$
q^\ast = \argmin_{q} L_{\tau}(q)
$$


% --------------------------------------------------------------------------------
%                                SGD for Quantile Estimation
% --------------------------------------------------------------------------------
\section{Quantile Estimation Using SGD}

\subsection{SGD for Loss function}

Let $q_0$ be the initial guess of quantile estimate. 
By SGD, the estimate is updated each step with a data point from the distribution.
$$
q_{k+1} = q_k - \alpha_k g_k
$$
where $ \alpha_k $ is a suitable stepsize and 
$$
g_k = \partial L_{\tau}^{(k)}(q_k) \in \frac{\partial l_\tau(x_k - q_k)}{\partial q_k}
$$ 
\textbf{Notice: partial is taken because the gradient of a single variable function
euqals the partial of it}
\\\\
Then we have
$$
q_{k+1} = 
    \begin{cases}
        q_k + \alpha_k \tau               & x_k - q_k > 0\\
        q_k - \alpha_k (1-\tau)           & x_k - q_k \leq 0\\
        % [\tau, -(1 - \tau)] & t = 0
    \end{cases}
$$

\subsection{Pseudo Code for the Algorithm}



    
\end{document}