\algnewcommand{\LineComment}[1]{\State \(\triangleright\) #1}
% New definitions
\algnewcommand\algorithmicswitch{\textbf{switch}}
\algnewcommand\algorithmiccase{\textbf{case}}
\algnewcommand\algorithmicassert{\text{assert}}
\algnewcommand\Assert[1]{\State \algorithmicassert(#1)}%
% New "environments"
\algdef{SE}[SWITCH]{Switch}{EndSwitch}[1]{\algorithmicswitch\ #1\ \algorithmicdo}{\algorithmicend\ \algorithmicswitch}%
\algdef{SE}[CASE]{Case}{EndCase}[1]{\algorithmiccase\ #1}{\algorithmicend\ \algorithmiccase}%
\algtext*{EndSwitch}%
\algtext*{EndCase}%

% \graphicspath{{Figures/Multi/P2/}{./}} 

\chapter{Simultaneous Multiple Quantile Estimation}
\label{ch: multi_quant}

\graphicspath{{Figures/Multi/}{./}} 

When the estimation for only a single quantile value is not enough, the idea of multi-quantile estimation appears. 
In this chapter we introduce the problem of simultaneous multi-quantile estimation along with two related methods. This chapter is organised as follow:

Section~\ref{sec: multi_intro} introduces the multi-quantile estimation for streaming data, followed by a discussion on the basic ideas to attack this problem. The next two sections will show how it is solved by methods focusing on different aspects.

Section~\ref{sec: multi_shiftQ} shows the \textit{shiftQ} algorithm for simultaneous quantile estimation by implementing the idea similar to SGD.

Section~\ref{sec: multi_{p2}} shows the \textit{$P^2$} algorithm that solves the problem in a much different way.

Section~\ref{sec: multi_discussion} list the comparison between the two methods, the discussion about the problem and our conclusion.

\section{The problem and opportunity of multi-quantile estimation}
\label{sec: multi_intro}

In real life implementations, quantile estimation usually do not focus on a single quantile value. For example, a common request is to show the median (the $0.5$-quantile) of the distribution, and at the same time show the outlier boundaries at ends of the distribution, like the $0.1$- and $0.9$- quantiles. It is also highly possible that multiple quantile estimates are required for a data distribution analysis.

A trivial solution for multi-quantile estimation is to run multiple single quantile estimation processes in parallel, such that each quantile is estimated by one process. The multi-process solution however, leads to two main problems:
\begin{enumerate}
    \item The parallel processing capacity of the implementation software/hardware.
    \item The missing information from the monotone property of quantiles.
\end{enumerate}

The following two multi-quantile estimation methods can simultaneously estimate multiple quantile numbers in one process with the utilization of the non-decreasing feature. Both algorithms, \textit{shiftQ}\cite{hammerJointTrackingMultiple2019b} and \textit{extended $p^2$}\cite{raatikainenSequentialProcedureSimultaneous1993} utilize the monotone property by ensuring a positive distance between one quantile and its previous quantile, such that 
$q_{i=} - q_{i-1} > 0$. The algorithm explanation of the shiftQ algorithm is described in section \ref{sec: multi_shiftQ}, and the next section \ref{sec: multi_{p2}} for the extended $p^2$ algorithm.

% ----------------------------------- shiftQ ---------------------------------------

\section{The shiftQ methods}
\label{sec: multi_shiftQ}
the shiftQ section

\subsection{Method Description}

1. \textbf{Overall}: The basic idea is to estimate a central quantile point, and update other quantiles based on the central one. 

2. \textbf{Motivation}: The difference $diff$ between the two quantiles for the $x_{n}$ observation is:
$$
diff = |Q_x(q_{k+1}) - Q_x(q_k)|
$$
    Note that $diff > 0$ for all time, so different quantiles never cross each other. This property is guaranteed by the update function DUMIQE() and its restriction that the input quantile estimate must > 0.

3. \textbf{Update one quantile}: The difference is calculated based on the idea of "shift distribution". Let $X$ denotes the original distribution of data stream, and let distribution $Y$ denotes a shifted version of $X$ such that $Y = X + constant$. In this way, the quantile $q_{k+1}$ can be updated by implementation of shifting. The basic steps are:
\begin{itemize}
    \item Calculate $Q_x(q_k)$, which is the shift constant
    \item Get shift observation $y_{k+1} =  Q_x(q_k) - x$
    \item Calculate the new quantile $Q_y(q_{k+1})$ in $Y$
    \item Shift the change back to $X$: $Q_x(q_{k+1}) = Q_y(q_{k+1}) + shiftConstant$
\end{itemize}

4. \textbf{Update the bigger quantiles:} Starting from central quantile $q_c$, the estimates for $q_{c+1}, ..., q_{K}$ are calculated one based on another. So each time step 2 is repeated from $q_c$ to $q_{K-1}$

5. \textbf{Update the smaller quantiles:} Similar to step 3, for smaller quantiles, estimates for $q_{c-1}, ..., q_{1}$ are calculated similarly one based on another.

\subsection{The shiftQ Algorithm}
\begin{algorithm}
    \caption{The shiftQ Algorithm}\label{alg:multi_shiftQ}
        \begin{algorithmic}[1]
            \Require{Dataset $X$ (with positive data points only), Demanding quantile values $[\tau_1, \tau_2, ..., \tau_P]$}
            \Ensure{Demaning quantile estimates $[\tau_1\text{-}q, \tau_2\text{-}q, ..., \tau_P\text{-}q]$}
            % \Procedure{frugal}{$X,\tau$}            \Comment{X is the dataset}
            % \State {$d = 0, y_i = 0$ for $i = 1, 2, ..., n$}           \Comment{Default initialization}
            % \For{$k = 0,1,...$ \textbf{do}}                  %\Comment{Parameter update for each input data point}
            %     \State {Sample $i$ from {$1,2,...,n$}}
            %     \State {$d=d - y_i + f^{\prime}_i(x))$}
            %     \State {$y_i = f^{\prime}_i(x)$}
            %     \State{$x = x - \frac{\alpha}{n}d$}
            % \EndFor
            % \State{\textbf{end for}}
        \end{algorithmic}
\end{algorithm}
\subsection{Experiment Results}

\begin{figure*}[h!]
	\includegraphics[width=1\columnwidth]{shiftQ/shiftQ_vs_SGD.png}
	\caption{Comparison between shiftQ and SGD}
\end{figure*}

% ----------------------------------- Extended P2 ---------------------------------------

\section{The Extended $P^2$ Algorithm}
\label{sec: multi_{p2}}

The \textit{Extended $P^2$} algorithm\cite{raatikainenSequentialProcedureSimultaneous1993} uses exactly the same idea of the $P^2$ algorithm\cite{jainP2AlgorithmDynamic1985}. So for a better understanding, we introduce the $P^2$ algorithm first in \ref{subsubsec: description_{p2}}, then the generation method for the extended $P^2$ algorithm.
In section \ref{subsec: algo_extended_{p2}}, the detailed algorithm for extended $P^2$ is provided, followed by its experiment results in section \ref{subsec: exp_extended_{p2}}.

\subsection{Method Description}

\subsubsection{The $P^2$ Method}
\label{subsubsec: description_{p2}}

The intuition of the $P^2$ method is the assumption that any three adjacent quantiles form a parabolic formula.
Specifically, the method interprets quantile estimation as a relationship based on quantile values and their ordering positions, and applies either linear or parabolic adjustments based on the information from their neighbours.

Suppose the straightforward quantile computation for [$\tau_1, ..., \tau_M$] that sorts all observations from the dataset $X = \{x_i\}^N_1$. Let $x_i$ denote the $i$th smallest value of $X$, then we have $x_1 < x_2 < ... < x_N$. 
To find the $\tau_i$-quantile of $X$, we need to find the data observation at \textit{marker position} $m_i = \tau_i \cdot N$. Given the marker position, we then retrieve the value of the marker, the corresponding \textit{quantile value} $q_i = x_{m_i}$. 
For 3 adjacent quantiles at $\tau_{i-1}, \tau_i, \tau_{i+1}$, their quantile values can be independently computed in the same way, as shown in Figure \ref{fig: {multi_relationship_p2}}.

\begin{figure*}[h!]
	\includegraphics[width=0.8\columnwidth]{P2/relationships.png}
    \caption{Relationship between $\tau$ value, marker position and quantile values correspondingly for 3 adjacent quantiles}
    \label{fig: {multi_relationship_p2}}
\end{figure*}

For quantile estimation, however, storing and sorting the entire dataset is infeasible. The $P^2$ method records only information only about demanding quantile values, and update them on the arrival of new observations. The update method, based on different conditions, is either a \textit{Piecewise-Parabolic} ($P^2$) formula, or a linear formula.

The information recorded for each quantile contains 3 parts: the marker position, the desired marker position, and the quantile value. For quantile $\tau_i$ with current observation number $M$, the desired marker position is $m_i \prime = 1 + (N-1)\tau$. This algorithm aims at keeping the marker position $m_i$ close to its desired position $m_i \prime$ while new observation coming in. The quantile value estimation $q_i$ is then updated accordingly after the current marker position is fixed. Figure \ref{fig: {multi_parabolic_p2}} demonstrates the update.

\begin{figure*}[h!]
	\includegraphics[width=1\columnwidth]{P2/parabola.png}
    \caption{Quantile value update using the Piecewise-Parabolic($P^2$) formula}
    \label{fig: {multi_parabolic_p2}}
\end{figure*}

The quantile value update is based on the Piecewise-Parabolic assumption that any three adjacent makers form a parabolic curve of the form $q_i = aq_i^2 + bq_i + c$.

If a marker is moved $d$ positions to the right, the new quantile value is updated by:
\begin{align}
\begin{split}
    q_{i} \leftarrow q_{i}+ & \frac{d}{n_{i+1}-n_{i-1}}\\
    \cdot & { \bigg[ 
        \left(n_{i}-n_{i-1}+d\right) \frac{\left(q_{i+1}-q_{i}\right)}{\left(n_{i+1}-n_{i}\right)}
        +\left(n_{i+1}-n_{i}-d\right) \frac{\left(q_{i}-q_{i-1}\right)}{\left(n_{i}-n_{i-1}\right)}
        } \bigg] \\
    n_{i} \leftarrow n_{i}+&d \text{ where } d = \pm 1
\end{split}
\end{align}

When the parabolic assumption cannot be applied to the quantile value update, the altherative update is a piecewise-linear formula:
\begin{equation}
    \begin{array}{l}
    q_{i}=q_{i}+d \frac{\left(q_{i+d}-q_{i}\right)}{n_{i+d}-n_{i}} \\
    n_{i}=n_{i}+d
    \end{array}
\end{equation}

In short, the update method for $\tau_i$-quantile takes only 2 steps:\\
When a new observation comes,
\begin{enumerate}
    \item Update marker position $m_i$ to approach the desired marker position $m_i \prime$.
    \begin{enumerate}
        \item If the quantile value is bigger than the coming observation, move the marker position by one position to the right.
        \item Adjust the marker position again if it is more than one position away from the desired marker. The move $d$ is $1$ if the marker position moves right, and $-1$ for moving left.
    \end{enumerate}
    \item Update the quantile value
    \begin{enumerate}
        \item Try applying the parabolic update to the quantile value. Note it might change the non-decreasing order of quantiles, that is, $q_i \geq q_{i+1}$.
        \item If the ordering of quantile values is changed by the parabolic update, try the linear update instead.
    \end{enumerate}
\end{enumerate}


\subsubsection{The Extended $P^2$ Method}
In $P^2$, the update of a quantile relies on its position relative to the neighbour markers, indicating the accuracy of neighbour marker positions is important. 
It sets one marker for each quantile value, that is, a total of $M$ markers for [$\tau_1, ..., tau_m$].
The extended $P^2$ algorithm improves the estimation accuracy by the introduction of "middle markers",  which nearly doubles the amount of markers in $P^2$. 
It means, in the initialization part, there will be $2M+3$ markers for
$$
\tau = 0, \frac{0+\tau_1}{2}, \tau_1, \frac{\tau_1 + \tau_2}{2}, \tau_2, ..., \tau_{M}, \frac{\tau_M+1}{2}, 1
$$
And the estimation update for all those markers follows the exact rule in the $P^2$ algorithm.

The only difference between extended $P^2$ and $P^2$ is the extension of markers at the initialization stage. At a doubly expensive computation cost, the quantile estimation reaches a higher accuracy from the extra information, as shown in the work of \citeauthor{raatikainenSequentialProcedureSimultaneous1993}\cite{raatikainenSequentialProcedureSimultaneous1993}. Besides the extra information brought by the extra markers, the extended $P^2$ algorithm also benefits from a better initialization by a larger sampling. The larger sampling in extended $P^2$ reduces the possibility of severely unevenly distributed initializations, which becomes important for  $P^2$, as it is sensitive to the initialization of quantiles.
% In $p^2$, only desired quantiles have their information recorded and used for 

\subsection{The Algorithm of Extended $P^2$}
\label{subsec: algo_extended_{p2}}

\begin{algorithm}
    \caption{The $P^2$ Algorithm}\label{alg:multi_p2}
        \begin{algorithmic}[1]
            \Require{Dataset $X$, $M$ Demanding quantile values $[\tau_1, \tau_2, ..., \tau_M]$}
            \Ensure{Demaning quantile estimates $[\tau_1\text{-}q, \tau_2\text{-}q, ..., \tau_M\text{-}q]$}
            \State
            \State{\textbf{A. Initialization}}
            \State {The first M observations (sorted): \{$x_1,x_2,...,x_M$\}}
            \For{$(i = 1, ..., M-1)$}
                \State {Marker height:    $q_i = x_i $}
                \State {Marker position:   $m_i = i$}
                \State {Desired Marker position:  $m_i\prime = (m-1)\tau_i + 1$}
            \EndFor

            \State
            \State{\textbf{B. For each new observation $x_j$, $j \geq M+1$, perform the following}}
            \Switch{$s$}            \Comment {Find the cell $k$ such that $q_k \leq x_j < q_{k+1}$}
                \Case{$x_j < q_1$}
                    \State{$q_1 = x_j, k = 1$}
                \EndCase
                \Case{$q_i \leq x_j < q_{i+1}$}
                    \State {$k = i$}
                \EndCase
                \Case{$q_M < x_j$}
                    \State {$q_M = x_j, k = M-1$}
                \EndCase
            \EndSwitch
            \State
            
            \State{$m_i = m_i + 1$; $i = k+1, ..., M$}       \Comment{Increment some marker positions}
            % \Comment{Different from that on the $P^2$ paper}
            \State{$m_i\prime = m_i\prime + \tau_i$}            \Comment{Update all the desired positions}

            \State
            \State{Adjust marker heights $2$ to $M-1$ if necessary:}
            \For{$i = 2, 3, ..., M-1$}
                \State {$d_i = m_i\prime - m_i$}
                \If {($ d_i \geq 1 \text{ and }  m_{i+1} - m_i > 1$) or 
                     ($ d_i \leq -1 \text{ and }  m_{i-1} - m_i < -1$) }
                    \State {$d_i = sign(d_i)$}
                    \State {$q_i\prime = \text{parabolic}(q_1)$}     \Comment{Try the $P^2$ update}
                    \If {$ q_{i-1} < q_i\prime < q_{i+1}$}
                        \State {$q_i = q_i\prime$}
                        \Else                           \Comment{Else use linear update}
                            \State{$q_i = \text{linear}(q_i)$}
                    \EndIf
                    \State {$m_i = m_i + d_i$}          \Comment{Update marker position}
                \EndIf
            \EndFor

            \State
            \State {\textbf{C. Return quantile estimates} }     
            \LineComment{The result is available after any number of observations}
            \State {$[\tau_1\text{-}q, \tau_2\text{-}q, ..., \tau_M\text{-}q] = [q_1, q_2, ..., q_M]$}
        \end{algorithmic}
\end{algorithm}

\begin{algorithm}
    \caption{Extended $P^2$ Algorithm}\label{alg:multi_ext_p2}
        \begin{algorithmic}[1]
            \Require{Dataset $X$, Demanding quantile values $[\tau_1, \tau_2, ..., \tau_M]$}
            \Ensure{Demaning quantile estimates $[\tau_1\text{-}q, \tau_2\text{-}q, ..., \tau_M\text{-}q]$}

            \State
            \State {\textbf{0.Extend number of markers from $M$ to $2M+3$}}
            \LineComment{Evenly fill the intervals between $0,1$ and each 2 quantile values}
            \State{$[\tau_1\prime, \tau_2\prime, ..., \tau_{2M+3} \prime] = [0, \frac{0+\tau_1}{2}, \tau_1, \frac{\tau_1 + \tau_2}{2}, \tau_2, ..., \tau_{M}, \frac{\tau_M+1}{2}, 1]$}

            \State
            \State{\textbf{1. Apply $P^2$ with the extended initialization}}
            \LineComment{And returns an estimate of extended list}
            \State{$[q_1\prime, ..., q_{2M+3}\prime] = P^2$ ($X$, $[\tau_1\prime, \tau_2\prime, ..., \tau_{2M+3} \prime]$)}
            \State
            \State {\textbf{2. Return quantile estimates} } 
            \State {Extract the quantile estimates for the original $M$ quantile values}
            \State {$[\tau_1\text{-}q, \tau_2\text{-}q, ..., \tau_M\text{-}q] = [q_3\prime, q_5\prime, ..., q_{2M+1}\prime]$}
        \end{algorithmic}
\end{algorithm}

% 0, 0.5, 1, 1.5, 2, ..., M,    (M+1)/2, 1
% 1, 2,   3, 4,   5, ..., 2M+1, 2M+2,   2M+3            

\subsection{Experiment Results}
Toy data experiment only
\label{subsec: exp_extended_{p2}}

% \begin{figure*}[h!]
% 	\includegraphics[width=1\columnwidth]{P2/P2.png}
% 	\caption{The P2 algorithm for Gaussian (mean = -20, std = 1)}
% \end{figure*}

% \begin{figure*}[h!]
% 	\includegraphics[width=1\columnwidth]{P2/P2_std_20.png}
% 	\caption{The P2 algorithm for Gaussian (mean = -20, std = 20)}
% \end{figure*}

% \begin{figure*}[h!]
% 	\includegraphics[width=1\columnwidth]{P2/P2_std_001.png}
% 	\caption{The P2 algorithm for Gaussian (mean = -20, std = $0.001$)}
% \end{figure*}

\section{Discussion and Conclusion}
\label{sec: multi_discussion}